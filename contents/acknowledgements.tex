\hspace*{2em}我谨以此文,向所有在此研究过程中给予我支持与启迪的人致以最深切的谢意。

致安东尼达斯导师
感谢我的导师安东尼达斯大法师。您的智慧如同达拉然紫罗兰之塔的辉光,指引我在魔法伦理与历史溯源的迷雾中寻得方向。从斯坦索姆事件的伦理学重构到霜之哀伤的精神寄生模型,您的批判性思维训练让我学会用奥术棱镜折射真相。尤其感谢您在冰冠堡垒符文分析中提供的《古代统御魔法图谱》,它让耐奥祖的操控术无所遁形。

致达拉然的同窗与战友
感谢罗宁议长与卡德加阁下。在塞拉摩的深夜讨论中,你们对时间线修正的建议(尽管卡德加总爱用“我曾见过一万种可能”作为开场白)为论文注入了青铜龙军团的宏观视角。同时,感谢我的助手金迪·火花,你整理的《天灾军团行动日志》和咖啡因药剂支撑了我无数个通宵。

致普罗德摩尔家族与库尔提拉斯
父亲戴林·普罗德摩尔,尽管我们因理念分歧而刀刃相向,但您教会我的责任感始终铭刻于心。母亲凯瑟琳,您的信笺跨越无尽之海送至诺森德前线,那句“真理需以理性而非怒火捍卫”成为我研究中的锚点。

致白银之手骑士团与肯瑞托
感谢乌瑟尔·光明使者生前留下的训练手札,您的圣光箴言与阿尔萨斯的堕落笔记形成残酷的镜像,为认知失调理论提供了不可替代的原始数据。达拉然图书馆的禁书区管理员,请原谅我“借用”了《通灵学派密卷》——我以六人议会之名保证,那些被暗影腐蚀的书页已妥善净化。

致无名勇士与时空见证者
感谢那些参与“阿尔萨斯意识残片回溯实验”的肯瑞托志愿者,你们承受的梦境低语与冰霜反噬,为巫妖王双重人格理论提供了实证。另感谢青铜龙克罗米,尽管您坚持“历史不可剧透”,但那次在时光之穴的“偶然”迷路,确实让我瞥见了斯坦索姆未被污染的平行时空。

最后,致阿尔萨斯·米奈希尔
这篇论文或许是你我故事的句点,亦是艾泽拉斯的警示碑。在冰冠冰川的寒风中,我仍能听见那个高喊“为洛丹伦而战”的年轻圣骑士的回声。愿这份研究让后人明白:真正的力量,不在于统御众生之冠,而在于直面人性深渊的勇气。

——吉安娜·普罗德摩尔
于塞拉摩废墟重建前夕