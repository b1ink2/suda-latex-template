\newpage
\linespread{1.5}

\begin{center} % 设置小二号字体和黑体
	{\heiti \zihao{-2} \textbf{圣光之子的陨落:论阿尔萨斯∙米奈
希尔的堕落与巫妖王身份的跨维度构建}}
\end{center}

\begin{center} % 设置小四号宋体,文本加粗,且“摘要”两个字中间空两个空格
	{\songti \zihao{4} \textbf{摘~  要}} 
\end{center}

{
\zihao{-4}

  本文以亲历者视角,结合白银之手骑士团档案、达拉然魔法伦理研究及艾泽拉斯历史学理论,系统性分析阿尔萨斯·米奈希尔从圣骑士向巫妖王转变的驱动力。通过解构斯坦索姆事件、霜之哀伤的精神寄生、耐奥祖的意志侵蚀等关键节点,论证其堕落并非偶然的“英雄黑化”,而是外部压力、制度缺陷与个体认知异化的多重共振结果。本研究旨在为艾泽拉斯道德哲学与权力心理学提供新的范式参考。

{\songti \zihao{-4} \textbf{关键词:}}巫妖王;堕落机制;圣骑士伦理;制度缺陷;认知异化
% 关键词3-5个用中文分号(;)隔开
}


\newpage

\begin{center}
    \zihao{-2}
    \textbf{The Fall of the Light's Champion: On the Corruption of Arthas Menethil and the Cross-Dimensional Construction of the Lich King's Identity}
\end{center}

\begin{center}
    \zihao{4}
    \textbf{Abstract}
\end{center}

{
\zihao{-4}

  This paper systematically analyzes the driving forces behind Arthas Menethil’s transformation from a paladin to the Lich King through the firsthand perspective of Jaina Proudmoore, integrating archival records from the Order of the Silver Hand, Dalaran’s research on magical ethics, and historical theories of Azeroth. By deconstructing pivotal events such as the Stratholme Purge, the psychological parasitism of Frostmourne, and Ner’zhul’s ideological erosion, it argues that his corruption was not a mere "heroic downfall" but a multidimensional resonance of external pressures, systemic flaws, and individual cognitive alienation. The study aims to provide a new paradigmatic framework for moral philosophy and the psychology of power within the context of Azeroth.

\textbf{Keywords:} Lich King;Mechanism of Corruption;Paladin Ethics;Systemic Flaws;Cognitive Alienation
}


