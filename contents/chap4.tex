\chapter{结论:制度缺陷与人性异化的交互作用}

\hspace*{2em}阿尔萨斯的堕落史可归结为三阶段模型:1)道德异化(斯坦索姆的“必要之恶”合理化);2)认知劫持(霜之哀伤对执念的寄生性利用);3)体制同化(天灾军团意志对个体身份的吞噬)。

这一过程揭示了艾泽拉斯社会治理的两大隐患:1)圣骑士制度的伦理局限性:过度强调牺牲与绝对正义,缺乏对极端情境的应对指南;2)权力监督机制的缺失:洛丹伦王室未能建立对王子的制衡体系,导致其独断行为失控。

作为亲历者,我呼吁达拉然与白银之手建立联合伦理委员会,以防止未来再度出现“以崇高之名的暴君”。阿尔萨斯的悲剧并非个人的失败,而是整个时代对人性复杂性认知不足的代价。

