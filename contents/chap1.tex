\chapter{道德绝对主义崩塌:斯坦索姆事件的认知重构}
\section{瘟疫危机中的伦理困境}
当斯坦索姆居民感染不可逆的亡灵瘟疫时,阿尔萨斯面临的并非传统伦理学中的“电车难题”,而是圣骑士誓言的内在矛盾:保护无辜者与消灭威胁源之间的冲突\cite{silverhand2020}。乌瑟尔与我的反对并非出于懦弱,而是基于圣光教义中“生命神圣性不可剥夺”的核心原则\cite{antonidas2015}。然而,阿尔萨斯将我们的质疑视为“软弱”,转而拥抱结果主义暴力——通过屠杀实现“更高正义”。

\section{认知重构:从守护者到裁决者}
阿尔萨斯对斯坦索姆的决策标志着其身份的根本转变。通过白银之手骑士团训练记录可发现,他在此事件后频繁引用“必要之恶”为暴力辩护,并开始质疑乌瑟尔的教导。这种转变符合认知失调理论\cite{festinger1957}:当行为与原有信念冲突时,个体会修改信念以合理化行为。阿尔萨斯通过将感染者“非人化”(称其为“活尸”而非“病人”),消解了屠杀的道德负担。

